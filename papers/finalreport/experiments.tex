\section{experiments}

\subsection{Running Gem5}
The gem5 simulator can be run in two modes: Syscall Emulation mode and
Full system mode.

The Systemcall Emulation mode is useful to run and analyze simple programs that
perform tasks involving traditional system calls. And the program to simulate
has to be statically linked because gem5 does not link executables dynamically.
As the name suggests, this mode is only useful to test user mode programs with
no modifications to the kernel. Since we need to add new instructions that deal
with user level interrupts where we queue packets for other cores, this mode
was not that much of a help.

The FullSystem mode as the name implies takes a kernel image and a disk image
that has filesystem, boots the kernel and can be used to execute the programs
from the mounted filesystem. This mode can simulate a complete system that 
involves interrupt mechanism which is needed for our implementation of the
stacklets and user level interrupts. 

Gem5 supports simulating various CPUs, caches and also two CPI models,
out of order and in-order pipelined model. We perform all the micro and macro 
benchmarks, building kernel on x86 infrastructure. For these experiments, we 
built a linux kernel, from the source with the smp support so that we can test 
upto 255 cores. We use a Linux VM to compile the benchmarks to form binaries, 
which are manually copied into the disk image that gets mounted internal to 
gem5.

\subsection{Micro benchmarks}






\subsection{Macro benchmarks}

