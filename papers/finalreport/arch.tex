\section{Architectural Modifications for User-Level Interrupts}\label{sec:arch}
We have made a few architectural modifications in order to realize user-level
interrupts. At the highest level we had to introduce a new interrupt type in
addition to the existing interrupts in the X86 64-bit architecture. User-level
interrupts in our current implementation are treated as non-maskable interrupts,
and in essence cannot be deferred and must be handled as soon as possible.
Interrupts are checked during the fetch phase of the out-of-order pipeline
simulated in Gem5. If a user-level interrupt has been generated, once it has
been propogated to the intended core, a \textit{pendingNonMaskableInt} flag
conveys that there is a non-maskable interrupt waiting to be handled. Following
this, the type of the non-maskable interrupt is checked. \textit{pendingULI}, a
flag introduced by us in the code-path for handling interrupts denotes the
existence of a user-level interrupt waiting to be handled. Using the mechanism
described below the user-level interrupt is serviced.

\subsection{ULI Instructions}
The following are the new instructions that were added to X86 in Gem5 for being
able to implement user-level interrupts. The subsections below describe the
functionality of the instructions in detail and their exact signatures and
opcodes can be found in Table \ref{tab:uli_instructions}.

\subsubsection{TOGGLEULI}
TOGGLEULI is used to toggle between enabling and disabling user-level
interrupts on a given core. Rather than implement separate enable user-level
interupts and disable user-level interrupts, we implemented one function that
toggle's the state based on the argument passed to it.

\subsubsection{SENDI}
This instruction is the main instruction driving user-level interrupts. On
receiving this instruction we create a message to be send using the X86
inter-processor interrupt mechanism (described in detail in Sec
\ref{sec:network}). SENDI sends a message atomically to the intended core. This
means that the message sending is instantaneous and not scheduled at a later
time. The response of the message is received in the \textit{recvResponse}
routine. Currently we assume reliable and correct message delivery, and hence,
ignore the response. We might need to perform differently in case the SENDI
mechanism involves responding with a NACK.

\subsubsection{RETULI}
The RETULI instruction is executed by the user-level interrupt handler when it
finishes executing its body, just before returning. The behavior of this
function is similar to \textit{iret} in the traditional interrupt handling
context. RETULI is entrusted with restoring the state of the processor to what
it was doing before it was interrupted. This currently involves resetting the
PC, SP and RDI to their original values before executing the handler.

\subsubsection{GETCPUID}
It is crucial for multi-core applications intending to use user-level interrupts
to know which core they are currently running on. This is facilitated by the
user code calling the GETCPUID function. This is a convenience function that can
be removed if other mechanisms exist to query the core on which the current
thread is running.

\subsubsection{SETUPULI}
Since user-level interrupts are in user land, unlike traditional kernel-level
interrupts, they need stack space in order to execute their handler. This stack
space has to be in user-space to correctly match the privilege level of the
application and for other security reasons. We envision that any application
using user-level interrupts with start with invoking SETUPULI on every core,
passing it as argument the pointer to a sizable allocated memory chunk (in our
example, 4KB). Since this memory chunk is malloc'd by the user, it belongs to
the virtual address space of the application itself. This enables us to set the
SP to this value whenever a user-level interrupt needs to be handled, making it
save all its handler state not just in the context of the application, but also
in user-space.

\begin{table}[]
\centering
\caption{ULI Instructions}
\label{tab:uli_instructions}
\begin{tabular}{lll}
\hline
\textbf{Instruction} & \textbf{Opcode} & \textbf{Signature}                                                      \\ \hline
TOGGLEULI          & 0x55            & uint64\_t stacklet\_uli\_toggle(uint32\_t enable, uint16\_t flags)      \\
SETUPULI             & 0x56            & uint64\_t stacklet\_setupuli(uint64\_t stackaddr)                       \\
SENDI                & 0x57            & uint64\_t stacklet\_sendi(void *callback, void *p, uint16\_t dest\_cpu) \\
GETCPUID             & 0x58            & uint64\_t stacklet\_getcpuid()                                          \\
RETULI               & 0x59            & uint64\_t stacklet\_retuli()
\end{tabular}
\end{table}

\subsection{Inter-Processor Interrupt Communication}\label{sec:network}
The Gem5 simulator has a basic inter-processor network infrastructure in order
to support inter-processor interrupts (IPIs). We leaverage this framework for
sending user-level interrupts across cores as well. The infrastructure is
essentially a message-passing interface (MPI). The sending core needs to
construct a message to send an interrupt to a different core other than itself.
The message was originally a 32-bit union, and for all existing purposes (other
than user-level interrupts) was self-contained, i.e. it could hold the type of
the interrupt that needed to be invoked on the receiving core along with the
necessary arguments. The arguments involved in SENDI are not small enough to fit
in a 32-bit union. We could have possibly increased the size of the message by
several bytes by cramming the SENDI arguments in the message itself, keeping the
current self-contained nature of messages. But, rather than making the message
itself bulky, we chose to increase it by just 4 bytes and made it a 64-bit
union. Along with the existing fields, we added a 32-bit attribute in the
message called \textbf{global\_message\_map\_key}. This field is a key of a
global hash table which maps 32-bit keys to a custom structure that holds all
the SENDI arguments that we intended to pass to the receiving core. This
workaround can easily be realized in real architectures by storing the arguments
temporarily in unused registers till the interrupt was communicated to the
receiving core, which can then copy it to its local cache regions as it sees
fit. The receiving core, on receiving an interrupt from a different core
inspects the message to understand the type of interrupt that it needs to
generate. If the interrupt is a user-level interrupt, it extracts the key from
the message and looks up the structure in the \textbf{msg\_map}, from where it
gets the arguments as sent by the remote core. We push the received user-level
interrupt in a queue of similar interrupts, which is decribed in more detail in
section \ref{sec:uli_queue}.

\subsection{Custom Registers}\label{sec:regs}
As mentioned above, the state of the process currently executing on the core
handling the user-level interrupt must be saved in order to restore it after
having serviced the interrupt. Currently we are saving the PC, SP and RDI, but,
we can imagine additions to this going ahead. For this purpose we would need as
many registers as the number of registers that could get affected in handling
the interrupt. As of now, we are storing the state in variables that are a part
of the currently executing thread context. The job of these variables is the
same, i.e. save state for future restoration.

\subsection{Queuing ULIs}\label{sec:uli_queue}
User-level interrupts, unlike regular non-maskable interrupts, need not be
received one at a time. Moreover, we cannot necessarily ignore an incoming
user-level interrupt from a different core because we are servicing one at this
time. It depends on which user-level interrupts are disabled (using the
TOGGLEULI instruction mentioned above). But, we can very well imagine
user-level interrupts being generated for multiple applications simultaneously,
each of which must be handled somewhere down the line. In order to facilitate
this requirement, we have a ULI queue in Gem5. Incoming user-level interrupts,
when detected on receiving cores, are added to a per-core ULI queue. Interrupts
are serviced from the front of the queue, i.e. in the order that they were
received.
