\section{Introduction}

There is no denying that we need parallelism and we should try harder for 
exploting parallelism in the modern day programs. But just by using some
thread library on every task and protecting a shared resource with lock
would not give parallelsim to the programs. The thread mechanism we use in
present day programs does not really know if all the processors are busy or
some of them are idle. They just create threads and lock the protected
resources to have mutual exclusion. This would really be a problem when the
programs spend most of their times in communicating among the threads, locking
and unlocking resources. We need a way to identify the current exten of 
parallelism in the system and create the threads only when needed, thereby
preventing unnecessary performance drop due to the inter-core communication.

In this paper, we solve this problem by using the concept called \textit{Stacklets}.
Stacklets are just like traditional stacks along with negligible book-keeping
information to link other stacklets. This mechanism simulates the behaviour of
a sequential call. Each thread starts like a sequential call just like a normal
procedure has been called. When there is a need of work for any other processor
or if the current procedure blocks, then we create a new stacklet and let the
needy processor steal the work after the current procedure. We call such
procedure calls \textit{Parallel ready sequential calls}. This mechanism
creates new parallel threads only when needed thereby reducing the overhead
involved, had the program used normal threads. With the stealing of one
processor's work by other, we make sure that the parallel readysequential call
setup cost is less than that of the eager thread setup mechanism. The overhead
involved is minimal as there is only 16 bytes added to the stacklet and
overflow of the stacklets being detected by a simple calculation isntruction.

Another thing we emphasize in this paper is the inter-core communication
involved when the above mentioned stealing mechanism happens. Usually, any
inter-core communication would require the intervention of kernel, setting up
interrupt arguments and finally switching to user mode to finish the stolen
work. But this mechanism might nullify the benefits of using Stacklets as it
is obviously a big overhead on every switch of the mode. To this end, we make
use of User Level Interrupts, to send the message to the other core, without
switching to the kernel mode. Using gem5 simulator, we simulated these
instructions, so that all the communication happen in the user mode. The sender
code just inserts a packet into destination core's queue and the receiver,
on the next iteration of timer, sees if there is any work and executes it
accordingly.

