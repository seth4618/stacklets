%
% Copyright (c) 1991 by ucb
%
% Permission to use, copy, modify, and distribute this file
% in any form (electronic, hardcopy, etc.) for any purpose is
% hereby DENIED without express permission of author.
%
% Author:               
% Version:              84
% Creation Date:        Fri Apr  2 02:40:10 1993
% Filename:             c-code.tex
% History:
%       SCG     2       Fri Apr  2 02:40:16 1993
%               made ` and ' really act liek { and } in beginC & endC
%               added progref, program enviornment, fixed blank lines
%               in beginC/endC etc.
%

% ----------------------------------------------------------------
% CODE FONT text (e.g. {\cf x := 0}).
% \cf is like \tt, but small and declares underscore to be
% an ordinary character.
%\renewcommand{\cf}{\small\tt\catcode`\_=12\catcode`\%=12\catcode`\~=12{}}

% \ttbraces makes \{ and \} generate brace characters.  Use in \tt font.
\def\ttbraces{\def\{{\relax\char`\{}\def\}{\relax\char`\}}}

% ----------------------------------------------------------------
% EXAMPLE environment taken from LAKTEX, but with singlespacing and
% indentation.
\newdimen\exampleindent
\exampleindent\leftmargini
\let\oldexample\example
%\def\example{\oldexample\singlespacing\leftskip\exampleindent}
\def\example{\oldexample\leftskip\exampleindent}

%\examplepenalty=10000

% ----------------------------------------------------------------
% CEXAMPLE environment, like example environment but in code font and
% allowing { and }
\newenvironment{cexample}{\begin{example} \cf\ttbraces}{\end{example}}

\def\ttc{\cf\ttbraces}
% ----------------------------------------------------------------
% VERBATIM, but ...???
%\makeatletter
%\def\verbatim{\small\@verbatim \frenchspacing\@vobeyspaces \@xverbatim}
%\makeatother

% ----------------------------------------------------------------
% C CODE DISPLAYS.
% C code displays are enclosed between \beginC and \endC.
% Most characters are taken verbatim, in typewriter font,
% Except:
%  Commands are still available (beginning with \)
%    but use ` and ' instead of { and }
%  Math mode is still available (beginning with $)
%
% scg - changed \topsep from 0 to .1in
% scg - changed \leftmargin from 0 to 2.0em
% scg - hacked so blank lines do the right thing

\outer\def\beginLC{%
  \def\par{\leavevmode\endgraf} 
  \begin{list}{$\bullet$}{%
    \setlength{\topsep}{0in}
    \setlength{\parskip}{0in}
    \setlength{\partopsep}{0in}
    \setlength{\itemsep}{0in}
    \setlength{\parsep}{0in}
    \setlength{\leftmargin}{0in}
    \setlength{\rightmargin}{0in}
    \setlength{\itemindent}{0in}
  }\item[]  \catcode`\{=12
  \catcode`\}=12
  \catcode`\&=12
  \catcode`\#=12
  \catcode`\%=12
  \catcode`\~=12
  \catcode`\_=12
  \catcode`\^=12
  \catcode`\~=12
  \catcode`\!=12
  \catcode`\'=2
  \catcode`\`=1\obeyspaces\obeylines\footnotesize\tt}

%  \catcode`\'=12
%  \catcode`\`=12


\outer\def\endLC{%
  \end{list}
  }

\newlength{\minilinelen}
\setlength{\minilinelen}{\textwidth}
\addtolength{\minilinelen}{-6em}

\outer\def\beginC{%
\hspace{4em}\begin{minipage}[t]{\minilinelen}
\vspace{-\abovedisplayskip}
    \begin{list}{$\bullet$}{%
    \setlength{\topsep}{1ex}
    \setlength{\parskip}{0in}
    \setlength{\partopsep}{0in}
    \setlength{\itemsep}{0in}
    \setlength{\parsep}{0in}
    \setlength{\leftmargin}{4em}
    \setlength{\rightmargin}{0in}
    \setlength{\itemindent}{0in}
  }\item[]  \catcode`\{=12
  \catcode`\}=12
  \catcode`\&=12
  \catcode`\#=12
  \catcode`\%=12
  \catcode`\~=12
  \catcode`\_=12
  \catcode`\^=12
  \catcode`\~=12
  \catcode`\!=12
  \catcode`\'=2
  \catcode`\`=1\obeyspaces\def\par{\leavevmode\endgraf}%
\obeylines\footnotesize\tt}

%  \catcode`\'=12
%  \catcode`\`=12


\outer\def\endC{%
  \end{list}
%  \vspace{-1.5ex}
  \end{minipage}
  }

% WHy is this here? {\obeyspaces\gdef {\ }}

% ----------------------------------------------------------------
% CODE environment for producing C code with the following capabilities
%
%       lines are numbered on the left, starting from 1
%       \cs will be recognized by tex as the command \cs
%               the active chars {} are replaced by `'
%       \llabel{foo} will have \ref{foo} produce a reference to the line #
%       \linelabel{foo} is short for \llabel{line-foo} and can be referenced
%       with \lref{foo}.  (see also \proglref, \lineref}
%       tabs will generate an error, so replace tabs with spaces.
%
%       If you want centering and such, put it in a minipage of 0in width
%
%       This is not really a verbatim environment.  The heart of this
%       is a tabbing enviroment, so \=. \+, etc % should be used with 
%       care. (but \' and \` produce ' and ` respectively
%
%       example use:  
%       \begin{code} 
%       foo()
%       {
%           int a_b = 1 % 2;    \linelabel{decl}
%       }
%       \end{code}
%       In the above text, \lineref{decl} declares the variable {\cf a_b}.
%
%       KNOWN BUG: '#' can not be the first character!
%
\makeatletter

%
% define the length of the numbers on the left
%
\newlength{\codenumwidth}
\setlength{\codenumwidth}{3em}

%
% define the into and outof verbatim character codes
%
\def\my@verb{\catcode`\{=12
  \catcode`\}=12
  \catcode`\&=12
  \catcode`\#=12
  \catcode`\%=12
  \catcode`\~=12
  \catcode`\_=12
  \catcode`\^=12
  \catcode`\~=12
  \catcode`\!=12
  \catcode`\'=2
  \catcode`\`=1
}

\def\my@reg{\catcode`\{=1
  \catcode`\}=2}

%
% define the counters and \refsetcount, which does the set usually
% done in \refstepcounter (which won't work here do to how tabbing
% defines things
%
\@definecounter{progline}
\def\progline{\refstepcounter{progline}}

\def\refsetcounter#1{\let\@tempa\protect
\def\protect{\noexpand\protect\noexpand}%
\edef\@currentlabel{\csname p@#1\endcsname\csname the#1\endcsname}%
\let\protect\@tempa}

%
% define the line label commands and \lref
%
{\catcode`\{=12 \catcode`\}=12 \catcode`\'=2 \catcode`\`=1
\gdef\llabel{#1}`\label`#1'\showdefhere`#1''
\gdef\linelabel{#1}`\label`line-#1'\showdefhere`line-#1'''  % brace hack }
\def\lref#1{\ref{line-#1}\showref{line-#1}}

%
% number lines and adjust the reference counter
%
\def\my@adv{\\ \refstepcounter{progline} \theprogline\>\refsetcounter{progline}\obeyspaces\my@verb {}}
\def\my@advnonum{\\ \refstepcounter{progline} \>\refsetcounter{progline}\obeyspaces\my@verb {}}

{\gdef\hacki{\catcode`\^^I=13}
% turn on objeylines so we can fool around with ^^M 
% EVERY LINE MUST END WITH A %
%
\makeatletter \obeylines%
%
% define the code environment
%
% touch next character -- eliminate any leading ^^M 
\gdef\code{\@ifnextchar.\mYcode\mYcode}% 
\gdef\coden{\@ifnextchar.\mYcoden\mYcoden}% 
%
% the real definition of code is here
%
\gdef\mYcode{%
%redefine end so it turns {} back on.
\let\my@end=\end\def\end{\my@reg\@ifnextchar.{\my@end}{\my@end}}%
%remove any added space above display
\vspace{-\abovedisplayskip}%
%make ^^M active and set it to \mytabcr
\obeylines\let^^M\mytabcr%
%make us small and in tt font
\footnotesize\tt%
%reset line counter
\global\setcounter{progline}{0}%
%define \' and \` to insert ' and `
\def\@tablab{'}\def\@tabrj{`}%
%make tab an active char with no defintion, so it gives us an error
\hacki%
%put us in tabbing mode with room for the line numbers on left
\tabbing \hbox to \codenumwidth{} \= \kill \my@adv}%
%
% define the end-code environment (which just takes us out of tabbing)
\gdef\endcode{\endtabbing}%
% this is ^^M, check for \end so last line has no \\ 
% and otherwise do an \my@adv
\gdef\mytabcr{\@ifnextchar\end{}{\my@adv}}
%
% the real definition of code (no number) is here
%
\gdef\mYcoden{%
%redefine end so it turns {} back on.
\let\my@end=\end\def\end{\my@reg\@ifnextchar.{\my@end}{\my@end}}%
%remove any added space above display
\vspace{-\abovedisplayskip}%
%make ^^M active and set it to \mytabcr
\obeylines\let^^M\mytabcrn%
%make us small and in tt font
\footnotesize\tt%
%reset line counter
\global\setcounter{progline}{0}%
%define \' and \` to insert ' and `
\def\@tablab{'}\def\@tabrj{`}%
%make tab an active char with no defintion, so it gives us an error
\hacki%
%put us in tabbing mode with room for the line numbers on left
\tabbing \hbox to 0em{} \= \kill \my@advnonum}%
%
% define the end-code environment (which just takes us out of tabbing)
\gdef\endcoden{\endtabbing}%
% this is ^^M, check for \end so last line has no \\ 
% and otherwise do an \my@adv
\gdef\mytabcrn{\@ifnextchar\end{}{\my@advnonum}}}
\makeatother

% ----------------------------------------------------------------
% labels are references for the program environment and code

\newcommand{\progref}[1]{Program~\ref{prog-#1}\showref{prog-#1}}
\newcommand{\lineref}[1]{Line~\lref{#1}}
\newcommand{\linerefs}[2]{Lines~\lref{#1}--\lref{#2}}
\newcommand{\proglref}[2]{Line~\ref{prog-#1}.\lref{#2}\showref{prog-#1}}
