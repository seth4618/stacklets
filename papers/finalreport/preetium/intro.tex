\section{Introduction}
Majority of the current computer systems embrace Multicore Architecture for its
suitability in running applications with inherent parallelism.
The parallelism can be extracted by the hardware or can be made explicit by the
software. Some examples of the former are Superscalar and Simultaneous
MultiThreading class of processors. Compilers can expose parallelism through
techniques such as loop unrolling or taking hints from the programmer about
parallel sections of the program such as in OpenMP and Cilk frameworks.
Programmers can explicitly create parallel threads through system calls such as
fork() or by using pthread libraries. However parallel applications bring
in a set of challenges which are typically not seen during sequential execution.
This paper explains how User Level Interrupts can be used to handle some of them.

One of the primary challenges in writing parallel applications is to synchronize
concurrent access to shared data structures. Several locking primitives have
been proposed in the past to handle synchronization. These primitives either
use shared variables or message passing primitives to synchronize between
processors. This paper proposes to use the best of both these worlds with User
Level Interrupts or ULIs. The motivation comes from the fact that ULIs prevent
traps into the kernel while still maintaining consistency of shared data
structures. This helps shift synchronization handling entirely to user space,
thus giving more flexibility in implementing it in user space,
and saving on context switch overheads. The paper implements two locking schemes
using ULIs, which is then evaluated on Gem5 simulator. The simulator emulates
the core features of interrupt handling such as sending, disabling and enabling
of user level interrupts, thereby allowing us to evaluate our scheme against the
traditional locking mechanisms.

The next part of this paper addresses eager parallelism. Programs are said
to have eager parallelism when they spawn a thread for each piece of code that
is independent in itself, such as a function call. This works well if there are
as many processors as there are threads. Otherwise the overhead of creating,
running and destroying these threads is unnecessary when they are eventually
going to be run sequentially. It wastes precious memory and cpu capacity. This
issue was addressed with the lazy thread implementation of such programs wherein
threads are created only when there are idle cores, else the code executes
sequentially. Lazy threads use stacklets to allocate an execution frame to each
thread. Stacklets enable efficient memory management of threads by allocating
stack space incrementally rather than as a monolithic chunk. If more stack space
is required, an additional chunk of memory is created and linked with the
previous. Each time a new stacklet has to be allocated, the thread can trap into
the kernel which will allocate the same and resume the thread. Alternately when
the head of the stack is reached, a ULI can be generated to allocate a new
stacklet by a userspace function itself. Once again, this prevents overhead of a
context switch into the kernel. This paper sets the foundation for exploring
this use case of ULIs by implementing Lazy thread handling using stacklets and
evaluates a simple fibonacci program against the implementation.

This paper thus makes the following contributions:

1. Explores new synchronization primitives using ULIs and evaluates them against
the traditional locking primitives.

2. Implements Lazy thread handling for improving performance of eagerly parallel
applications.

3. Highlights the importance of stacklets for multithreaded applications and
notes the use of ULIs in its implementation.

4. Showcases ULIs as potential solution for handling functions in the user space,
which are traditionally handled by the kernel.

