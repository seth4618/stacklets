\section{Related Work}

There has been effort in the past to move I/O interrupt handling into userspace,
\cite{parker} as opposed to this paper which focuses on synchronous interrupts such as
interprocessor interrupts and memory overflow in the case of stacklets. 

This saves the overhead incurred due to two sources. One, context switch to
kernel and two, cache misses incurred due to operating system interrupt handling
routines. This is especially true in clustered systems where ethernet packets
can arrive at a phenomenal pace on high speed network interfaces. There are
concerns though about situations when interrupts cannot be immediately serviced
in userspace.  This could happen if the process for which the interrupt is
directed has to wait to be scheduled in. In such cases, interrupt handling can
be taken over by the kernel as usual.

As a part of the effort to move drivers into userspace, the Linux kernel
community has tried in the past to handle device interrupts in userspace \cite{website:lkml}. One of
the approaches has been for the userspace to block on a /proc interface whenever
interrupts occur. However, the control still switches to the kernel before
handing over interrupt handling to userspace, thereby still incurring the
overhead of a context switch into the kernel. Another approach enables userspace
to register for interrupt handling in the kernel. Here too, the kernel takes
over control first on an interrupt, before handing it over to userspace.

There have been efforts to handle parallelism in userspace using user level
threading libraries so as to keep off the overhead incurred due to kernel
interferrence. But as one can see, none of the above efforts address the issue
of aiding parallelism through user level interrupts.
