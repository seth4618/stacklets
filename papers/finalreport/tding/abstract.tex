\begin{abstract}

This is an age when multicore machines are ubiquitous and parallel programs are abundant.  Doing two things at a time must be faster than doing them in sequential. But do we always get expected speed-up? Unfortunately, Amdahl's law says this is not the case. Sequential components inside the parallel tasks can quickly become a bottleneck. Given this sad fact, we do not always need so many parallelism. In this paper, we proposed several techniques to address such problems. We not only show that excessive parallel calls can be downgraded into parallel-ready sequential calls when necessary, but still achieve the same performance as pure sequential calls using stacklets, but also introduce our user-level interrupt design. By cleverly leveraging user-level interrupt, we claim to reduce thread level synchronization overhead and achieve higher throughput. We demonstrate this using two types of ULI-based queuing locks and show their improved performance.

\end{abstract}
