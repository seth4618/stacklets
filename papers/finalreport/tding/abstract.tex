\begin{abstract}

This is an age when multicore machines are ubiquitous and parallel programs are abundant.  Doing two things at a time must be faster than doing them in sequential. But do we always get expected speed-up? Unfortunately, Amdahl's law says this is not the case. Given the sad fact, do we really need so much parallelism? In this paper, we proposed several techniques to address such problems. We not only show that excessive parallel calls can be downgraded into parallel-ready sequential calls when necessary, but still achieve the same performance as pure sequential calls using stacklets, but also introduce our user-level interrupt design. It can also be used to reduce synchronization overhead and achieve higher thread throughput. We demonstrate this using 2 types of ULI-based locks with their performance.

\end{abstract}
